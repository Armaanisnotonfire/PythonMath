\documentclass[10pt]{article}
% generated by Madoko, version 0.9.23-beta
%mdk-data-line={1}

\usepackage[heading-base={2},section-num={False},bib-label={True}]{madoko2}
\usepackage{pstricks}
\usepackage{pst-plot}


\begin{document}



%mdk-data-line={16}
\mdxtitleblockstart{}
%mdk-data-line={16}
\mdxtitle{\mdline{16}Mathematics class}%mdk

%mdk-data-line={19}
\mdxsubtitle{\mdline{19}Written in python}%mdk
\mdxauthorstart{}
%mdk-data-line={24}
\mdxauthorname{\mdline{24}Enzo Mallard}%mdk

%mdk-data-line={27}
\mdxauthoraddress{\mdline{27}IUT-Nantes INFO 1 G4-2}%mdk

%mdk-data-line={30}
\mdxauthoremail{\mdline{30}enzo.mallard@etu.univ-nantes.fr}%mdk
\mdxauthorend\mdtitleauthorrunning{}{}\mdxtitleblockend%mdk

%mdk-data-line={18}
\begin{abstract}%mdk

%mdk-data-line={19}
\noindent\mdline{19}The aim of this library is to implement several mathematics aspects.
Currently, there are only the\mdline{20}~\mdref{sec-matrix}{Matrix}\mdline{20} class, the\mdline{20}~\mdref{sec-complex}{Complex}\mdline{20} class and the\mdline{20}~\mdref{sec-ratio}{Ratio}\mdline{20}.%mdk
%mdk
\end{abstract}%mdk

%mdk-data-line={24}
\section{\mdline{24}1.\hspace*{0.5em}\mdline{24}Matrix}\label{sec-matrix}%mdk%mdk

%mdk-data-line={25}
\noindent\mdline{25}This class contain 9 methods (and the \mdline{25}\_\mdline{25}\_\mdline{25}init\mdline{25}\_\mdline{25}\_\mdline{25} one), they will be explained in their sub-sections.%mdk

%mdk-data-line={27}
\subsection{\mdline{27}1.1.\hspace*{0.5em}\mdline{27}Initialisation}\label{sec-initialisation}%mdk%mdk

%mdk-data-line={28}
\begin{enumerate}%mdk

%mdk-data-line={28}
\item{}
%mdk-data-line={28}
\mdline{28}In this part, there are two assertions, this is the starting assertion, they must be true to create the instance :%mdk

%mdk-data-line={30}
\begin{itemize}[noitemsep,topsep=\mdcompacttopsep]%mdk

%mdk-data-line={30}
\item\mdline{30}The type of what define the matrix must be a list.%mdk

%mdk-data-line={31}
\item\mdline{31}Non-empty matrix, if there are not any row or if the row contain nothing (no columns) the assertion is false,
here we calculate the length by finding the max length of all the rows.%mdk
%mdk
\end{itemize}%mdk%mdk

%mdk-data-line={34}
\item{}
%mdk-data-line={34}
\mdline{34}After we have to add missing case, in fact, this class allow user to create an incomplete matrix like that :\mdline{34}\mdbr
\mdline{35}$\begin{pmatrix} 1 & 0 &   \\1 & 0 & 1 \\0 &   & \end{pmatrix}\qquad\text{ }$\mdline{35} so we want \mdline{35}$\text{ }\qquad\begin{pmatrix}1 & 0 & 0 \\1 & 0 & 1 \\0 & 0 & 0\end{pmatrix}$\mdline{35}.\mdline{35}\mdbr
\mdline{36}In the algorithm, we look for each row, and  we redefine the row as : \mdline{36}$\text{row = row + missing 0}$\mdline{36} the number of missing 0 is easily find because it\mdline{36}'\mdline{36}s the matrix length minus the row length.\mdline{36} \mdline{36}%mdk%mdk
%mdk
\end{enumerate}%mdk

%mdk-data-line={39}
\begin{enumerate}[,start=3]%mdk

%mdk-data-line={39}
\item{}
%mdk-data-line={39}
\mdline{39}Here, we need to define the main attribute, in this case, there are three attributes :%mdk

%mdk-data-line={41}
\begin{itemize}[noitemsep,topsep=\mdcompacttopsep]%mdk

%mdk-data-line={41}
\item\mdline{41}matrix.comp\mdline{41}\mdbr
\mdline{42}This is the main attribute, it contain all the matrix\mdline{42}'\mdline{42}s rows and store it as an array of arrays.%mdk

%mdk-data-line={43}
\item\mdline{43}matrix.h and matrix.l\mdline{43}\mdbr
\mdline{44}These attribute define the height (number of row) and the length (number of columns). 
They are not important but I\mdline{45}'\mdline{45}ve added them to enhance readability after.%mdk
%mdk
\end{itemize}%mdk%mdk
%mdk
\end{enumerate}%mdk

%mdk-data-line={48}
\subsection{\mdline{48}1.2.\hspace*{0.5em}\mdline{48}Representation}\label{mat-repr}%mdk%mdk

%mdk-data-line={49}
\noindent\mdline{49}This function is important in the class because it define the look of the class when it\mdline{49}'\mdline{49}s called.%mdk
\begin{mdpre}%mdk
\noindent~~{\mdcolor{navy}return}~{\mdcolor{maroon}"}{\mdcolor{gray}\textbackslash{}n}{\mdcolor{maroon}"}.join([{\mdcolor{maroon}"}{\mdcolor{gray}\textbackslash{}t}{\mdcolor{maroon}"}.join([str(n)~{\mdcolor{navy}for}~n~{\mdcolor{navy}in}~i])~{\mdcolor{navy}for}~i~{\mdcolor{navy}in}~self.comp])%mdk
\end{mdpre}\noindent\mdline{53}This 
  \mdline{54} \mdline{54} 

%mdk-data-line={55}
\section{\mdline{55}2.\hspace*{0.5em}\mdline{55}Complex}\label{sec-complex}%mdk%mdk

%mdk-data-line={58}
\begin{itemize}[noitemsep,topsep=\mdcompacttopsep]%mdk

%mdk-data-line={58}
\item\mdline{58}representation%mdk

%mdk-data-line={59}
\item\mdline{59}addition%mdk

%mdk-data-line={60}
\item\mdline{60}substraction%mdk

%mdk-data-line={61}
\item\mdline{61}negation%mdk

%mdk-data-line={62}
\item\mdline{62}multiplication%mdk

%mdk-data-line={63}
\item\mdline{63}transmutation (\mdline{63}\textbf{invert}\mdline{63})%mdk

%mdk-data-line={64}
\item\mdline{64}contain (python \mdline{64}\textquotedblleft{}in\textquotedblright{}\mdline{64} function)%mdk

%mdk-data-line={65}
\item\mdline{65}setitem (via matrix\mdline{65}[\mdline{65}(i,j)]=value)%mdk

%mdk-data-line={66}
\item\mdline{66}len ()%mdk
%mdk
\end{itemize}%mdk

%mdk-data-line={68}
\section{\mdline{68}3.\hspace*{0.5em}\mdline{68}Rational}\label{sec-ratio}%mdk%mdk

%mdk-data-line={71}
\begin{mdbmargintb}{4em}{}%mdk
\begin{mdflushright}%mdk
{\tiny\mdline{72}Created with~\href{https://www.madoko.net}{Madoko.net}.}%mdk
\end{mdflushright}%mdk
\end{mdbmargintb}%mdk%mdk


\end{document}
